\documentclass{article}

\usepackage{amssymb}
\usepackage{amsmath}
\usepackage{hyperref}

\usepackage{graphicx} % Required for inserting images
\usepackage{algpseudocodex}
% \usepackage[preprint]{neurips_2024}
\usepackage[preprint]{neurips_2024}
\begin{document}

\section{1.1 Exploratory data analysis}

\subsection{Basic facts}
We have a dataset that consists of 5471 (n) samples with 4124 (p) columns. The features of our data are all continuous, in a logarithmic scale and they expression levels 
for genes. It is worth mentioning that the dataset is sparse, that is, a lot of cells have many gene expressions that are 0. While it is hard to have an objective measure
of sparsity we can plot average gene expression levels across all our features, see Figure \ref{fig:hist_sparse} below, most genes are non zero for a limited number of cells.

\begin{figure}[h]

    
    \caption{Possible figure of average gene expression level across all genes. We can do either histogram or histogram by label.}\label{fig:hist_sparse}
\end{figure}


We have 2 distinct classes of cells, the TREG cells and the 
CD4+T cells. These are going to be our labels that we want to classify. We have some class imbalance, the ratio is 6/10 in favour of the CD4+T 
which is the dominant class. This is going to be important for the models we try to tune as some models have options to adjust for class imbalance.
We will also try the option of tuning the threshold decision for classifying to one cell or another. By default the probabilistic models in Scikit-learn
classify to the positive class if the conditional probability for the given model $\mathbb{P}(\textit{y}|X) > 0.5$\footnote{\href{https://scikit-learn.org/1.5/modules/classification_threshold.html}{see scikit references}}. 
We will tinker with this threshold to try to optimize the F1-Score given our class imbalance.  

\subsection{Visualization and dimensionality redcution}

Both the fact that $n \approx p$ and the fact that the data is sparse point us towards using towards using regularization and feature selection.
The instructions for this problem also make us use PCA with 10 components. To confirm whether the number of components is optimal, we can plot the cumulative sum of the explained variance by each of the components.
As you can see below in Figure \ref{fig:scree}, choosing just the first ten components makes us use a very amount of the total variance.
In Section \ref{sec:1.3} we will try to tune the number of components to get an improved F1-Score.

\begin{figure}[h]
    
    
    \caption{Scree plot: cumulative sum of PCA components.}\label{fig:scree}
\end{figure}

For completeness, we add Figure \ref{fig:tsne} where we use t-SNE to reduce the dimensionality of the data from 4124 to 2\footnote{ The underlying algorithm is stochastic 
and quite sensitive to how we tune the hyperparameter of perplexity} with the purpose of visualizing the joint distribution (after the t-SNE transformation) of both cells. 
Figure \ref{fig:tsne} shows some separability of the two classes. It also shows some cells of a given clas (TREG) in regions where the density is much higher for the other type of cell (CD4T), maybe we can use this as intuition as to why
in later sections we find it difficult to improve the F1-Score beyond 0.95.


\begin{figure}[h]
    
    
    \caption{Joint distribution of transformed data by t-SNE}\label{fig:tsne}
\end{figure}
% Make sure we explain why it doesn't work. -> T-sne or something like that.
% Correct way to use t-sne if we have space is to plot different perplexities. TRY DO IT AND JUST MENTION IT IN A NOTE.
% \section{Training baseline models and hypertuning}

\section{1.2}

\begin{itemize}
    \item Most tuning is done, VSM give very good results.
    \item Tough to do what shak is doing with pca \href{https://scikit-learn.org/1.5/auto_examples/compose/plot_digits_pipe.html}{Example from sklearn}

    \item RESULTS FROM GRIDSEARCH ARE NOT GOING TO BE REPLICABLE, RESULTS FROM FINAL ESTIMATOR BASED ON IT SHOULD BE.
\end{itemize}


\section{Our 3 models}\label{sec:1.3}

\begin{itemize}
    \item AdaBoost, just because we have seen it in class and it could be interesting.
    \item I wanted some classically strong classifier like random forest or GBDT?
    \item Play around with the PCA optimality. Or some other modern dimensionality reduction tool like t-SNE or UMAP. 
\end{itemize}

\section{}


\end{document}